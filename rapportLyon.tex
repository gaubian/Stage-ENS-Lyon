% Classe de document.
\documentclass[a4paper]{article}


%%% ENCODAGE ET LANGUES %%%
% Encodage de l'entrée.
\usepackage[utf8]{inputenc}

% Encodage de la police.
\usepackage[T1]{fontenc}

% Langues à charger. 
% \usepackage[francais]{babel}


%%% PAQUETS POUR LES MATHEMATIQUES %%%
\usepackage{amsmath}
\usepackage{amssymb}
\usepackage{amsthm}


%%% AUTRES PAQUETS %%%
\usepackage{graphicx}
\usepackage{hyperref}


\newtheorem{theorem}{Théorème}
\newtheorem{proposition}[theorem]{Proposition}
\newtheorem{lemma}[theorem]{Lemme}
\theoremstyle{definition}
\newtheorem{definition}{Définition}[section]
\theoremstyle{remark}
\newtheorem{remark}{Remarque}[section]
\newtheorem{example}{Exemple}[section]
\newtheorem{conjecture}{Conjecture}[section]


\newcommand{\itab}[1]{\hspace{0em}\rlap{#1}}
\newcommand{\tab}[1]{\hspace{.2\textwidth}\rlap{#1}}


\title{Majoration de la décomposition minimale d'un graphe régulier}
\author{AUBIAN Guillaume}

\begin{document}

\maketitle

\begin{abstract}
Voici le rapport de mon stage effectué à l'École Normale Supérieure de Lyon, dont l'objectif était de majorer la taille
de la décomposition minimale d'un graphe k-régulier en chemins disjoints, sous la direction de
Stéphan Thomassé
\end{abstract}

\part{Préambule}

\paragraph{}

\section{Encadrement au LIP}
Il s'agit du Laboratoire de l'Informatique du Parallélisme, situé au centre de Lyon. C'est le laboratoire d'informatique de l'École Normale 
Supérieure de Lyon, et la plupart des chercheurs y enseignent. J'étais dans l'équipe MC2 - pour Modèles de calcul, Complexité, Combinatoire -
qui s'intéresse de manière générale à la théorie de la complexité, à 
l'algorithmique sur les structures discrètes, et à la combinatoire de 
façon plus générale.


\paragraph{}
Le chercheur qui m'a encadré était Stéphan Thomassé, dont le domaine
de recherche est plutôt la théorie des graphes, même si j'ai aussi
beaucoup travaillé et communiqué avec d'autres chercheurs ou doctorants 
que je remercie chaleureusement , parmi lesquels Matthieu Rosenfeld, Aurélie Lagoutte et Éric Thierry.


\section{Le problème étudié}

\paragraph{}
Le problème que m'a proposé Stéphan Thomassé concerne donc la théorie 
des graphes, et possède l'énorme avantage d'être très facile à 
comprendre, ce qui m'a permis de m'y attaquer directement.


\paragraph{}
Tout d'abord, nous allons donner quelques précisions, notamment sur le vocabulaire spécifique.


\paragraph{}
Déjà, nous ne nous intéresserons qu'à des graphes simples non-orientés, i.e. avec au plus une arête entre chaque paire de sommets et aucune arête
d'un sommet vers lui-même. Un tel graphe est dit k-régulier si tous ses
sommets sont de degré k, i.e. ont k voisins.


\includegraphics{graphExample3.png}


\paragraph{}
Pour un tel graphe G, on nomme décomposition en chemins disjoints de G  un ensemble C de chemins de G tel que l'ensemble des sommets des chemins 
de C partitionne l'ensemble des sommets de G. On note
$\mu$(G)=min{|C|, C est une décomposition en chemins de G}.

\includegraphics{graphExample4.png}


\paragraph{}
La conjecture à démontrer est la suivante:
\begin{conjecture}
Si G est un graphe k-régulier à n sommets, $\mu(G) \leq \frac{n}{k+1}$
\end{conjecture}
C'est Cognont et Martin qui semblent l'avoir établie, en 2009 dans INSERER LE PAPIER CORRESPONDANT.


\paragraph{}
En 2014, R. Ravi a trouvé une $1 + O(\frac{1}{k})$-approximation en
temps polynomial ( et même linéaire ) du problème du voyageur du 
commerce dans le cas des graphes k-réguliers, à condition que cette 
conjecture soit vérifiée. Plus exactement, si la conjecture est vraie
pour une certaine valeur de k et de n, alors on a un algorithme en
temps linéaire d'approximation du problème du voyageur pour les graphes
k-réguliers à n sommets. C'est remarquable, vu que le problème du 
voyageur du commerce est NP-complet, même en le restreignant aux graphes
réguliers.


\paragraph{}
Quelques semaines avant le début de mon stage, Paul Seymour a fait
remarquer en conférence que dans le cas de graphes denses ( typiquement 
$k = \epsilon n$ ), cela donnait une $1 + O(\frac{1}{\epsilon n})$-approximation, et que ça semblait être un premier angle d'attaque 
possible. Mon maitre de stage s'étant rendu compte que le problème 
était beaucoup plus difficile que prévu et que c'était le plus 
intéressant, nous avons décidé de nous attaquer en priorité aux
graphes denses.

\paragraph{}
Notre plus grand résultat a été de le démontrer pour $k \geq \frac{n}{3}$, même si nous avons pu aborder plusieurs autres stratégies ( avec une
décomposition en anse, avec un arbre couvrant, avec une lollipop, le 
maximum ou minimum lexicographique ), qui n'ont malheureusement pas 
abouti, mais pourraient être approfondies.

\part{Les résultats obtenus}

\section{Les résultats directs}

\paragraph{}
Tout d'abord, on peut énoncer un certain nombre de résultats directs.


\paragraph{}
Dans un premier temps, on peut remarquer que la conjecture est immédiate si G admet un chemin hamiltonien: en effet, on a alors $\mu(G)=1$, ce qui vérifie forcément la conjecture vu que $k+1 \leq n$.


\paragraph{}
Or d'après le théorème de Dirac:
\begin{theorem}
Si le degré minimal d'un graphe G à n noeuds est supérieur ou égal à
$\frac{n}{2}$, alors G est hamiltonien
\end{theorem}


Donc, si $k \geq \frac{n}{2}$, la conjecture est vérifiée. À fortiori, la
conjecture est vérifiée sur les graphes complets ( vu que $k = n-1 \geq \frac{n}{2}$ ).


\paragraph{}
De plus, il est important de voir qu'on peut se limiter aux graphes connexes. En effet, si on peut partager un graphe G en deux composantes
$G_{0}$ et $G_{1}$ telles qu'il n'y ait aucune arête entre $G_{0}$ et $G_{1}$. On remarque que tout chemin de G est soit dans $G_{0}$, soit dans $G_{1}$.
Donc $\mu(G) \leq \mu(G_{0}) + \mu(G_{1})$. On a donc notre relation de
récurrence, et il suffit de résoudre le cas initial, qui est celui des graphes connexes.


\paragraph{}
En s'inspirant de ceci, il est évident que la conjecture est «sharp», puisqu'il suffit de considérer l'union de $\frac{n}{k+1}$ graphes complets à k sommets: chaque composante connexe possède un chemin 
hamiltonien d'après le théorème de Dirac, et donc il $\mu(G)=\frac{n}{k+1}$.


\includegraphics{graphExample5.png}


\paragraph{}
Que se passe-t'il pour des petites valeurs de k? On se restreint comme décrit plus haut à des graphes connexes.


\paragraph{}
Si k=0, alors le graphe est réduit à un sommet, i.e $\mu(G)=1=\frac{1}{0+1}=\frac{n}{k+1}$


\paragraph{}
Si k=1, le graphe n'est constitué que de deux sommets reliés par une arête, et donc possède un chemin hamiltonien, i.e. $\mu(G)=\frac{2}{1+1}=\frac{n}{k+1}$.


\paragraph{}
Si k=2, G est réduit à un cycle, i.e. $\mu(G)=1 \leq \frac{n}{3}$ car $n \geq k+1 = 3$.


\paragraph{}
Il y a aussi des preuves pour k=3, 4 et 5 dans le papier de Cognont et
Martin qui présente initalement la conjecture, mais on les redémontre facilement avec notre méthode du maximum lexicographique.

\section{La méthode de l'arbre couvrant}

\paragraph{}
Nous avons tout d'abord découvert deux théorèmes qui nous semblaient prometteurs pour la suite.

\paragraph{}
On notera $nb_{leaves}(T)$ le nombre de feuilles d'un arbre $T$. Le premier des deux théorèmes stipule:
\begin{theorem}
Soit $ST$ un arbre couvrant d'un graphe connexe G, alors $\mu(G) \leq nb_{leaves}(ST)$.
\end{theorem}

\begin{proof}
Il est évident que $\mu(G) \leq \mu(ST)$, vu qu'un chemin de $ST$ est un
chemin de G.

\paragraph{}
On va alors noter $PW(T)$ l'ensemble des chemins pointés d'un arbre $T$ que l'on définit par récurrence de la manière ainsi:

\paragraph{}
Si T est réduit à un sommet $r$ (qui est aussi sa racine), et en notant c le chemin réduit à $r$, $PW(T) = {c}$. On remarque que la racine de T (qui n'est autre que $r$), est à l'extrémité d'un chemin de $PW(T)$.

\paragraph{}
Sinon, si T est réduit à une racine $r$ avec $s$ fils $T_{1}, T_{2} ... T_{s}$. On choisit $c$ le chemin de $PW(T_{1})$ tel que la racine de $T_{1}$ est à l'extrémité de $c$. On définit alors $PW(T)$ comme étant l'union des $PW(T_{i})$ (pour $i$ allant de $1$ à $s$), si ce n'est qu'on rajoute $r$ à l'extrémité de $c$. On remarque alors qu'un des chemins de $PW(T)$ a pour extrémité la racine de $T$.

\paragraph{}
On démontre facilement par induction structurelle que $|PW(T)|= nb_{leaves}(T)$.


\paragraph{}
Et comme $PW(T)$ est un cas particulier de décomposition en chemins disjoints, $\mu(ST) \leq |PW(T)|$.


\paragraph{}
Donc $\mu(G) \leq \mu(ST) \leq |PW(T)| = nb_{leaves}(T)$
\end{proof}


\paragraph{}
On aimerait alors pouvoir démontrer qu'un graphe $G$ k-régulier à n noeuds admet toujours $O(\frac{n}{k})$ feuilles. Et le théorème 2 nous
permet de montrer que si notre conjecture est vraie, c'est le cas:
\begin{theorem}
Soit $G$ un graphe connexe, alors:

$\exists ST$ un arbre couvrant de G, $nb_{leaves}(ST)\leq 2\mu(G)$.
\end{theorem}

\begin{proof}
Soit $G$ un graphe connexe. On note $W= \{ w_{1};w_{2};...;w_{\mu(G)} \}$ une décomposition minimale de $G$ en chemins disjoints.

\paragraph{}
On note $G_{folded}$ le graphe replié de $G$, i.e. les sommets de $G_{folded}$ sont en bijection avec les chemins de $W$ et il y a une arête entre deux sommets de $G_{folded}$ si et seulement si il y a une arête
entre les deux sommets chemins correspondants de $W$.

\paragraph{}
$G$ étant connexe, et par construction de $G_{folded}$, il est connexe 
lui-aussi. On considère alors $ST_{folded}$, un arbre couvrant de $G_{folded}$. En «dépliant» $ST_{folded}$, on obtient un arbre couvrant $ST$ de $G$ (ceci se voit aisément, mais se démontre en remarquant que $ST_{folded}$ possède $\mu(G)-1$ arêtes, et que les chemins de $W$ utilisent $n - \mu(G)$ arêtes. Donc $ST$ utilise $n - \mu(G) + \mu(G)-1 = n-1$ arêtes. Comme $W$ partitionne les sommets de $G$, $ST$ est un arbre couvrant de $G$).
\end{proof}

Donc:

$\forall G, \exists ST,$ arbre couvrant de $G$, $ \mu(G) \leq nb_{leaves}(ST) \leq 2\mu(G)$

\paragraph{}
Il est à noter que ceci ne se restreint pas qu'aux graphes réguliers.

\includegraphics{graphExample6.png}

\paragraph{}
On doit alors démontrer que les graphes k-réguliers à n noeuds admettent
un arbre couvrant à $O(n/k)$ feuilles. On peut alors partir d'un arbre
couvrant avec un nombre minimal de feuilles, et montrer qu'il a forcément moins de $\frac{n}{k+1}$ feuilles, par exemple par l'absurde 
avec le théorème suivant.

\begin{theorem}
Soit $G$ un graphe connexe k-régulier, et $ST_{min}$ un arbre couvrant 
qui minimise $nb_{leaves}(ST_{min})$. Si $nb_{leaves} > \frac{n}{k+1}$,
alors deux feuilles de $ST_{min}$ ont un voisin commun dans $G$
\end{theorem}
\begin{proof}
Si, à chaque feuille, on associe ses $k$ voisins et elle-même - i.e. $k+1$ éléments - alors comme il y a plus de $\frac{n}{k+1}$ feuilles et $n$
sommets, on obtient presque notre résultat en utilisant le théorème des
tiroirs. En effet, il reste encore le cas de figure où deux feuilles sont
voisines: ceci est impossible, car si c'était le cas, on pourrait 
rajouter l'arête entre les deux, et supprimer une arête adjacente à un
noeud du cycle nouvellement créé de degré $\geq 3$ dans $ST_{min}$. 
Ainsi, on a supprimé deux feuilles pour en créer une, ce qui contredit
l'argument de minimalité.
\end{proof}

\paragraph{}
Dans la pratique, nous ne sommes pas arrivé à conclure malgré ce théorème. Néanmoins, on aimerait bien garder une structure qui permette
de conclure avec un théorème semblable.

\section{La méthode du maximum lexicographique}

\paragraph{}
C'est le cas avec cet angle d'attaque: ici, on considère une partition minimale de $G$ en chemins disjoints $W = \{w_{1};w_{2};...;w_{\mu(G)}\}$ (on demande donc que $|W|$ soit minimal), qui soit un maximum 
lexicographique parmi les partitions minimales en chemins disjoints de $G$. Pour rappel, cela signifie que pour toute partition minimale de $G$ 
en chemins disjoints $W'=\{w'_{1};w'_{2};...;w'_{\mu(G)}\}$ différente de $W$, $w_{1} > w'_{1}$ ou ($w_{1}=w'_{1}$ et $w_{2} > w'_{2}$) ou ($w_{1}=w'_{1}$ et $w_{2} = w'_{2}$ et $w_{3} > w'_{3}$) etc...

i.e. $\exists i \in [1;...;\mu(G)] \forall i' < i, (w_{i} > w'_{i}$ et $w_{i'}=w'_{i'})$

\begin{theorem}
Soit $x$ un sommet à l'extrémité d'un $w_{i}$, et $y$ un voisin de $x$
situé sur $w_{j}$, pour $i \neq j$. Alors, $y$ est à une distance supérieure ou égale à $w_{i}$ d'une extrémité de $w_{j}$.
\end{theorem}

\begin{proof}
Par l'absurde, on pourrait rajouter l'arête entre $x$ et $y$, et
supprimer l'arête partant de $y$ vers l'extrémité de $w_{j}$ la plus
proche. On obtient alors une contradiction puisque le nouveau chemin obtenu est plus grand que $w_{i}$ et que $w_{j}$.
\end{proof}


\paragraph{}
A fortiori, cela implique que les extrémités des $w_{i}$ ont tous leurs
voisins situés sur des $w_{i'}$ pour $i' \leq i$, comme on peut le voir 
sur la figure suivante:

\includegraphics{graphExample7.png}

\paragraph{}
Une autre conséquence est que si une extrémité $x$ de $w_{i}$ a un voisin sur $w_{j}$ pour $i \neq j$, alors $|w_{j}| \geq 2|w_{i}|$. On 
voudrait alors faire une chaîne $w_{i_{0}} \rightarrow w_{i_{1}} \rightarrow w_{i_{2}} \rightarrow ...$ tel que chaque élément soit deux 
fois plus grand que le précédent. Malheureusement, le cas de figure 
suivant semble être une barrière infranchissable:

\includegraphics{graphExample8.png}
\end{document}
